SOME GEOMETRIC APPLICATIONS OF THE DISCRETE HEAT FLOW

ALEXANDROS ESKENAZIS

Abstract. We present two geometric applications of heat flow methods on the discrete hyper-
cube {−1,1}n. First, we prove that if X is a finite-dimensional normed space, then the bi-Lipschitz
distortion required to embed {−1,1}n equipped with the Hamming metric into X satisfies

(cid:16){−1,1}n(cid:17) (cid:38) sup

cX

p∈[1,2]

n

Tp(X)min{n,dim(X)}1/p

,

where Tp(X) is the Rademacher type p constant of X. This estimate yields a mutual refinement
of distortion lower bounds which follow from works of Oleszkiewicz (1996) and Ivanisvili, van
Handel and Volberg (2020) for low-dimensional spaces X. The proof relies on an extension of
an important inequality of Pisier (1986) on the biased hypercube combined with an application
of the Borsuk–Ulam theorem from algebraic topology. Secondly, we introduce a new metric
invariant called metric stable type as a functional inequality on the discrete hypercube and prove
that it coincides with the classical linear notion of stable type for normed spaces. We also show
that metric stable type yields bi-Lipschitz nonembeddability estimates for weighted hypercubes.

2020 Mathematics Subject Classification. Primary: 46B85; Secondary: 30L15, 42C10, 46B07.
Key words. Hamming cube, Rademacher type, metric embeddings, Borsuk–Ulam theorem, stable type.

Let {−1,1}n be the n-dimensional discrete hypercube equipped with the Hamming metric

1. Introduction

(cid:88) i

=1

n

|x(i)− y(i)|,

(1)

1 2

ρ(x, y) =

∀ x, y ∈ {−1,1}n,

∀ p, q ∈ M,

where x = (x(1), . . . , x(n)) and y = (y(1), . . . , y(n)). The purpose of the present paper is to investi-
gate certain metric properties of the Hamming cube via heat flow methods.
1.1. Dimensionality of Hamming metrics. If (M, dM) is a metric space and (Y ,(cid:107) · (cid:107)Y ) is a
normed space, we say that M embeds into Y with bi-Lipschitz distortion at most D ∈ [1,∞),
if there exists a mapping f : M → Y satisfying

dM(p, q) ≤ (cid:107)f (p)− f (q)(cid:107)Y ≤ DdM(p, q).

(2)
The least D ≥ 1 for which such an embedding exists will be denoted by cY (M). The rapidly
growing field of metric dimension reduction aims to uncover conditions under which given
families of metric spaces admit (or do not admit) embeddings into low-dimensional normed
spaces with prescribed properties. Without attempting to survey this vast area, we note that
important contributions have been made on low-dimensional embeddings of finite subsets of
Hilbert space [29], arbitrary finite metric spaces [30, 3, 41, 42], discrete hypercubes [56, 34],
diamond graphs [14, 35, 52], Laakso graphs [24, 34], ultrametric spaces [8], series-parallel
graphs [15], recursive cycle graphs [1], Heisenberg-type metrics [32, 54], (cid:96)p variants of thin
Laakso structures [7, 9] and expander graphs [48, 50]. We refer to the survey [49] for more
bibliographic information and to [26, 38, 69, 2] for a sample of algorithmic applications.

The first to study the bi-Lipschitz embeddability of hypercubes into normed spaces was
Enflo. In the seminal work [20], he introduced the notion of roundness of a metric space and
used it to show that any embedding of the Hamming cube {−1,1}n into an Lp(µ) space, where
p ∈ [1,2], incurs bi-Lipschitz distortion at least n1−1/p (see also [19, 21] for additional early

1

n

(cid:16)

(cid:107)vi(cid:107)p
X,

(cid:88) i

=1

(cid:17)(cid:13)(cid:13)(cid:13)p

x(1), . . . ,−x(i), . . . , x(n)

Lp(µ) dσn(x) ≤ n(cid:88)
(cid:13)
(cid:13)
(cid:13)
(cid:13)
(cid:13)p
(cid:13)f (x)−f (−x)

results along these lines). More specifically, Enflo proved that if p ∈ [1,2], then any mapping
(cid:90)
of the form f : {−1,1}n → Lp(µ) satisfies the estimate
(cid:13)
(cid:13)
(cid:13)f (x)−f
Lp(µ) dσn(x), (3)
where σn is the uniform probability on {−1,1}n. This readily implies that if f has bi-Lipschitz
distortion D, then D ≥ n1−1/p. In the follow-up work [21], he raised an influential problem by
asking for which normed spaces (X,(cid:107)·(cid:107)X), inequality (3) is satisfied for X-valued functions f up
this requirement to linear functions f (x) =(cid:80)n
to a multiplicative constant T , independent of the choice of f or the dimension n. Restricting
(cid:13)
(cid:13)
(cid:13)
(cid:13)p

i=1 xivi, one recovers the necessary condition
dσn(x) ≤ T p

(4)
which must be satisfied for every n ∈ N and vectors v1, . . . , vn ∈ X. If a normed space X satisfies
(4), we say that X has Rademacher type p and the least constant T is denoted by Tp(X). Af-
ter decades of substantial efforts (see [13, 64, 53, 25, 22]), Ivanisvili, van Handel and Volberg
resolved Enflo’s problem in the breakthrough work [27] by proving the sufficiency of this con-
dition, namely that any normed space of Rademacher type p also has Enflo’s nonlinear type p.
Consequently, any bi-Lipschitz embedding of {−1,1}n into a normed space X of Rademacher
−1n1−1/p. We note in passing that,
type p incurs distortion at least a constant multiple of Tp(X)
conversely, a classical theorem of Pisier [57] implies that if X does not have type p for any
p > 1, then {−1,1}n embeds into X with bi-Lipschitz distortion at most 1 + ε, for any ε > 0.

(cid:13)
(cid:13) n(cid:88)
(cid:13)
(cid:13)

{−1,1}n

{−1,1}n

{−1,1}n

(cid:90)

(cid:90)

xivi

i=1

Independently of this line of research, the beautiful (but perhaps overlooked) work [56] of
Oleszkiewicz established a nonembeddability result for discrete hypercubes in the context of
dimensionality reduction. Following Ball, Carlen and Lieb [5], we say that a normed space is
p-uniformly smooth, where p ∈ [1,2], if there exists a constant S > 0 such that

i=1

X

∀ x, y ∈ X,

(cid:107)x(cid:107)p

X +(cid:107)y(cid:107)p

X

2

≤(cid:13)(cid:13)(cid:13)(cid:13) x + y

2

(cid:13)
(cid:13)
(cid:13)
(cid:13)p

X

(cid:13)
(cid:13)
(cid:13)
(cid:13) x− y

2

(cid:13)
(cid:13)
(cid:13)
(cid:13)p

X

+ Sp

;

(5)

cX

the least such constant S is denoted by Sp(X). A well-known tensorization argument due to
Pisier [60] shows that Tp(X) ≤ Sp(X), yet there exist examples of normed spaces X for which
Tp(X) < ∞ whereas Sp(X) = ∞ for p ∈ (1,2], see [61, 28, 65]. The main result of Oleszkiewicz’s
paper [56] asserts that the distortion required to embed {−1,1}n into a finite-dimensional
normed space X satisfies

(cid:16){−1,1}n(cid:17) ≥ sup
(see also [6] for a precursor of this result for linear embeddings). This substantially improves
the bound cX({−1,1}n) (cid:38) supp∈[1,2] Tp(X)
−1n1−1/p which follows from [27], at least for spaces X
with p-smoothness constant Sp(X) (cid:16) 1 and dimension dim(X) << n.
The first goal of the present paper is to revisit the technique used for Oleszkiewicz’s nonem-
beddability theorem [56], in particular proving the following mutual refinement of his result
and of the recent work of Ivanisvili, van Handel and Volberg [27].
Theorem 1. Let (X,(cid:107)·(cid:107)X) be a finite-dimensional normed space. Then, for any n ≥ 1, we have

Sp(X)min{n,dim(X)}1/p

p∈[1,2]

(6)

n

(cid:16){−1,1}n(cid:17) (cid:38) sup

cX

n

Tp(X)min{n,dim(X)}1/p

(7)

p∈[1,2]

We emphasize that, in contrast to Oleszkiewicz’s bound (6), Theorem 1 captures more accu-
rately the nonembeddability of the hypercube into (finite-dimensional subspaces of) normed
spaces which have Rademacher type p but are not r-smooth for any r ∈ (1,2], see [28, 65].

2

About the proof. Theorem 1 is proven by a combination of semigroup tools with a clever topo-
logical trick of [56]. More specifically, let f : {−1,1}n → X be a function, where X is a d-
dimensional normed space and d < n. An application of the Borsuk–Ulam theorem [43] for the
unique multilinear extension of f implies that there exists a subset σ ⊆ {1, . . . , n} with |σ| = d, a
product measure ν on {−1,1}σ and a point w ∈ {−1,1}σ c such that

f (x, w) dν(x) =

{−1,1}σ

{−1,1}σ

f (−x,−w) dν(x).

(8)

(cid:90)

(cid:90)

Then, a Poincaré inequality à la Enflo for the product measure ν (instead of the uniform mea-
sure σd) on the d-dimensional subcubes {−1,1}σ ×{w} and {−1,1}σ ×{−w} yields the distortion
bounds of Theorem 1 (see Theorem 14 and also equation (78) below).
In the case of p-uniformly smooth spaces, Oleszkiewicz [56] used (8) and a bootstrap argu-
ment for the Lipschitz constant of f , based on the two-point inequality (5), to obtain (6). In
our case, the biased Poincaré inequality which will yield (7) is an extension of an inequality
for the uniform measure that was proven in [27]. The key technical contribution of [27] was a
novel representation of the time derivative of the heat flow on {−1,1}n. Instead, we consider a
Markov process having the product measure ν as stationary measure (see Section 2) and prove
a formula for the time derivative of the corresponding semigroup (see Proposition 13) which
extends the formula of [27] (see also (47) below). Due to the fact that our product measure
ν is no longer the stationary measure of the random walk on a group, the resulting identity
lacks some homogeneity properties that were used in [27], but nevertheless it is sufficient for
the proof of the biased Poincaré inequality which is needed for our geometric application.
1.2. A Pisier–Talagrand inequality on the biased cube. For α ∈ (0,1) we denote by µα the α-
biased probability measure on {−1,1} with µα{1} = α and µα{−1} = 1−α. The proof of Theorem
1 yields as a consequence the following extension and refinement of Pisier’s inequality [64].
Theorem 2. For every p ∈ [1,∞) and α ∈ (0,1), there exist Kp,α,Cα ∈ (0,∞) such that the following
holds. For any normed space (X,(cid:107)·(cid:107)X) and any n ∈ N, every function f : {−1,1}n → X satisfies
(cid:13)
(cid:13)
(cid:13)
(cid:13)
(cid:13)f −

(cid:13)
(cid:13)
(cid:13)
(cid:13)
(cid:13)Lp(log L)p/2(µn

(cid:13)
(cid:13) n(cid:88)
(cid:13)
(cid:13)


(cid:90)






≤ Kp,α

(cid:13)
(cid:13)
(cid:13)
(cid:13)p

dσn(δ)

{−1,1}n

{−1,1}n

f dµn
α

δi(cid:128)α
i f

(cid:90)

Lp(µn

α;X)

α;X)

i=1

(9)

+ Cα(log n + 1)

dσn(δ).

If additionally X is assumed to be of finite cotype, then there exists Kp,α(X) ∈ (0,∞) such that





1/p

(cid:13)
(cid:13)
(cid:13)
(cid:13)
(cid:13)Lp(log L)p/2(µn

(cid:13)
(cid:13)
(cid:13)
(cid:13)
(cid:13)f −

≤ Kp,α(X)


(cid:90)






dσn(δ)

{−1,1}n

{−1,1}n

{−1,1}n

f dµn
α

δi(cid:128)α
i f

(cid:90)

Lp(µn

α;X)

α;X)

α;X)

δi(cid:128)α
i f

.

(10)

(cid:90)
(cid:13)
(cid:13) n(cid:88)
(cid:13)
(cid:13)

i=1

(cid:13)
(cid:13) n(cid:88)
(cid:13)
(cid:13)
(cid:13)
(cid:13)
(cid:13)
(cid:13)p

i=1







1/p
(cid:13)
(cid:13)
(cid:13)
(cid:13)L1(µn

The (standard) definitions of Orlicz norms, cotype and discrete derivatives will be given at
the main part of the article. Theorem 2 is an optimal vector-valued version of a deep loga-
rithmic Sobolev inequality of Talagrand [67]. In the case of the uniform measure which corre-
sponds to α = 1

2, Theorem 2 was proven recently in [16].

1.3. Metric stable type. Besides its interest in the context of embedding theory, the work [27]
of Ivanisvili, van Handel and Volberg was a major contribution to a long-standing research
program in nonlinear functional analysis called the Ribe program. Put simply, the foundational
idea of the program (as put forth by Bourgain [12], who was inspired by a landmark rigidity
theorem of Ribe [66]) is that isomorphic local properties of normed spaces can be equivalently
reformulated using only distances between pairs of points, with no reference to the spaces’
linear structure (see also [46, 4]). In this sense, the main result of [27] completes the Ribe
program for Rademacher type by proving that it is equivalent to nonlinear Enflo type.

3

Among the various isomorphic properties studied in the local theory of normed spaces,
stable type (see [58, 59]) plays a prominent role, in particular due to its relation to the problem
of identifying (cid:96)n
p subspaces of Banach spaces [63]. Recall that a symmetric random variable θ
is distributed according to the standard p-stable law, where p ∈ (0,2], if it satisfies Eexp(itθ) =
exp(−|t|p) for every t ∈ R. A Banach space (X,(cid:107) · (cid:107)X) is said to have stable type p if for every
(cid:18) n(cid:88)
r < p, every n ∈ N and every vectors v1, . . . , vn ∈ X, we have

(cid:19)1/p

(cid:18)

(cid:13)
(cid:13) n(cid:88)
(cid:13)
(cid:13)

i=1

E

(cid:19)1/r ≤ Cp,r(X)

(cid:13)
(cid:13)
(cid:13)
(cid:13)r

X

θivi

(cid:107)vi(cid:107)p

X

,

i=1

(11)

where the constant Cp,r(X) depends only on p, r and X, but not on n or the choice of v1, . . . , vn,
and θ1, θ2, . . . are independent standard symmetric p-stable random variables.
In the special case p = 2, p-stable random variables are Gaussians and thus stable type 2
coincides with Rademacher type 2. Moreover, any space with stable type p ∈ [1,2) also has
Rademacher type p but the converse does not necessarily hold. In fact, early results of Maurey
and Pisier in the theory of type (see [58, Proposition 3] and [59, Théorème 1]), show that a
normed space X has stable type p ∈ [1,2) if and only if X has Rademacher type p + ε for some
ε > 0. Therefore, formally, the Ribe program for stable type p ∈ [1,2) has been completed by
[27]: a normed space X has stable type p if and only if X has Enflo type p + ε for some ε > 0. In
other words, stable type p of a normed space X can be characterized by the validity of some, a
priori unknown, X-valued Poincaré-type inequality from a given family.

(cid:16)

(12)

2 dM

(cid:17)(cid:17)

.

(cid:107)w(cid:107)(cid:96)n
p,∞

To amend this, we shall introduce a new nonlinear invariant called metric stable type which
is again a Poincaré-type inequality for functions defined on the discrete hypercube and will
(cid:16)
serve as a metric characterization of stable type for normed spaces. If (M, dM) is a metric space
and f : {−1,1}n → M is an M-valued function, we denote
x(1), . . . ,−x(i), . . . , x(n)
f (x), f
r · #{i : |wi| ≥ r}1/p(cid:111)
(cid:110)
Moreover, we shall denote the weak (cid:96)p norm on Rn by

∀ x ∈ {−1,1}n,

∀ w ∈ Rn,

dif (x) def= 1

(cid:16)
(cid:17)p
f (x), f (−x)
{−1,1}n
p,∞ ≤ (cid:107) · (cid:107)(cid:96)n

def= sup
r≥0
where #S denotes the cardinality of a finite set S.
Definition 3. A metric space (M, dM) has metric stable type p ∈ (0,2) with constant S ∈ (0,∞) if for
(cid:90)
any n ∈ N, every function f : {−1,1}n → M satisfies

(cid:90)
Since (cid:107) · (cid:107)(cid:96)n
p by Markov’s inequality, any space with metric stable type p also has
Enflo type p with the same constant. We shall prove the following metric characterization.
Theorem 4. A normed space X has metric stable type p ∈ [1,2) if and only if X has stable type p.
Pisier’s K-convexity theorem [62], asserts that a Banach space X is K-convex if and only
if it has Rademacher type p for some p > 1. In view of the aforementioned characterization
of stable type [58, 59], this is equivalent to X having stable type 1 and thus we derive the
following metric characterization of K-convexity by a Poincaré inequality.

dσn(x) ≤ Sp

d1f (x), . . . , dnf (x)

p,∞ dσn(x).
(cid:96)n

(cid:17)(cid:13)(cid:13)(cid:13)p

(cid:13)(cid:16)
(cid:13)
(cid:13)

{−1,1}n

(13)

(14)

dM

,

Corollary 5. A normed space X is K-convex if and only if X has metric stable type 1.

Embeddings. The fact that stable type of Banach spaces is a refinement of Rademacher type
can also be seen at the level of (linear) embeddings. Indeed, if X is a space of Rademacher type
p, then standard considerations show that the Banach–Mazur distance between (cid:96)n
1 and any
n-dimensional subspace of X is at least a constant multiple of n1−1/p. This estimate is sharp
for X = (cid:96)p. On the other hand, if X has stable type p, then the characterization of [58, 59]

4

n

=1

1 2

(cid:88) i

∀ x, y ∈ {−1,1}n,

+, consider the metric space {−1,1}n

implies that X has Rademacher type r for some r > p. Hence, there exists ε(X) > 0 such that the
1 and any n-dimensional subspace of X is at least n1−1/p+ε(X).
Banach–Mazur distance between (cid:96)n
In the metric setting, Enflo type gives distortion lower bounds for embeddings of {−1,1}n.
More generally, given a vector w = (w1, . . . , wn) ∈ Rn
w which
is the hypercube {−1,1}n equipped with the weighted Hamming metric
wi|x(i)− y(i)|.

(15)
It follows readily from the definitions that if M has Enflo type p, then any embedding of {−1,1}n
into M incurs distortion at least a constant multiple of (cid:107)w(cid:107)(cid:96)n
1 /(cid:107)w(cid:107)(cid:96)n
w
p . For spaces of metric stable
type p, we have the following improvement for the distortion of weighted hypercubes.
(cid:17) ≥ (cid:107)w(cid:107)(cid:96)n
Proposition 6. If a metric space (M, dM) has metric stable type p with constant S, then
S(cid:107)w(cid:107)(cid:96)n
1
(16)
p,∞
p,∞ (cid:16) 1, whereas
−1/p, we have that (cid:107)w(cid:107)(cid:96)n
Concretely, choosing the weight vector w with wi = i
(cid:107)w(cid:107)(cid:96)n
(cid:16) (log n)1/p, therefore any embedding of {−1,1}n
w into a metric space of Enflo type p
incurs distortion at least n1−1/p/(log n)1/p instead of the asymptotically stronger lower bound
n1−1/p which one gets for target spaces of metric stable type p.
On the nonlinear Maurey–Pisier problem for type. A landmark theorem of Maurey and Pisier [44]
asserts that if pX ∈ [1,2] is the supremal Rademacher type of an infinite-dimensional Banach
space X, then X containts the spaces (cid:96)n
pX uniformly. In [63], Pisier used stable type to give a
much simpler proof of this important theorem, while simultaneously obtaining the following
quantitative refinement: for any ε > 0 and p ∈ [1,2), every normed space X containts a subspace
which is (1 + ε)-isomorphic to (cid:96)d
p provided that d is smaller than some explicit (unbounded)
function of ε and the stable type p constant of X. When X is infinite-dimensional, this yields
the result of [44] since X does not have stable type pX by the results of [58, 59].

(cid:16){−1,1}n

∀ w ∈ Rn
+,

In [13], Bourgain, Milman and Wolfson introduced a notion of nonlinear type, which is now
referred to as BMW type, and showed that if a metric space M does not have BMW type p for
any p > 1, then cM({−1,1}n) = 1 for any n ∈ N. Finding a satisfactory nonlinear Maurey–Pisier
theorem for metric spaces of supremal nonlinear type p > 1 remains an open problem (see also
[47, Section 6]). It would be interesting to understand whether the newly introduced notion
of metric stable type can lead to a nonlinear version of the result of [63]. We refer to Section 6
for further remarks and open problems which naturally arise from this work.

ρw(x, y) =

cM

w

.

p

Acknowledgements. I wish to thank Florent Baudier, Paata Ivanisvili and Assaf Naor for their
constructive feedback on this work.

2. Preliminaries

2.1. Probability. In this section, we outline the basics of analysis on the biased hypercube,
with an emphasis on the underlying semigroup structure.
The biased measure. Recall that, for α ∈ (0,1), the α-biased probability measure µα on {−1,1}
is given by µα{1} = α and µα{−1} = 1 − α. Moreover, if ααα = (α1, . . . , αn) ∈ (0,1)n, then we shall
denote by µααα the product measure µα1
}t≥0 on {−1,1} given by
The Markov process. For α ∈ (0,1), consider the transition matrices {pα
(1− e
1− (1− e

⊗···⊗ µαn on the hypercube {−1,1}n.

(cid:32)
1− (1− e
(1− e

(cid:33)
−t)(1− α)
−t)α

t (1,−1)
pα
pα
t (1,1)
t (−1,1) pα
t (−1,−1)
pα

−t)(1− α)
−t)α

∀ t ≥ 0,

(17)

(cid:32)

(cid:33)

=

t

5

(cid:17)

(cid:16)

(cid:89) i

=1

n

Moreover, for ααα = (α1, . . . , αn) ∈ (0,1)n consider the corresponding tensor products {pααα
{−1,1}n given by

t

}t≥0 on

∀ x, y ∈ {−1,1}n,

pαi
t

t

t

.

pααα
t (x, y) =

x(i), y(i)

As each pαi
2n × 2n matrices pααα
t
chain {Xααα

(18)
is a row-stochastic 2×2 matrix with nonnegative entries, the same holds also for the
}t≥0 is the transition kernel of a time-homogeneous Markov
t . Therefore, {pααα
P(cid:110)
}t≥0 on {−1,1}n, that is
∀ t, s ≥ 0,
}t≥0 be a Markov process on {−1,1}n with transition kernels

where x, y ∈ {−1,1}n. We shall need the following simple facts for this process.
Lemma 7. Fix ααα ∈ (0,1)n and let {Xααα
{pααα
Proof. Due to the product structure of the Markov chain, it suffices to consider the case n = 1,
that is, to prove that for α ∈ (0,1),

}t≥0 is stationary and reversible with respect to µααα.

}t≥0. Then, {Xααα

(cid:12)
(cid:12)
(cid:12) Xααα

t (x, y),

t+s = y

s = x

= pααα

(19)

Xααα

(cid:111)

t

t

t

t

f

.

(cid:105)

∈S

µαi

f d

Xααα
t

(22)

µαi .

0 = x

{−1,1}S

t (y, x).

(cid:111)(cid:90)

t (x, y) = µα(y)pα

(cid:89) i

(20)
This follows automatically by the expression (17) for the transition matrix. The simple fact
(cid:3)
that reversibility implies stationarity is well-known [37, Proposition 1.20].
}t≥0 has a simple probabilistic interpretation which we
The stationary Markov process {Xααα
shall now describe. For i = 1, . . . , n, let {Nt(i)}t≥0 be n independent Poisson processes of unit rate
0 is sampled from µααα independently of {Nt}t≥0. Then, at any time t > 0 for
and suppose that Xααα
which the process Nt(i) jumps for some i ∈ {1, . . . , n}, the corresponding value Xααα
t (i) is updated
independently from µαi . An explicit calculation shows that this probabilistic construction
gives rise exactly to the transition kernel of (17) and (18).
}t≥0 be the Markov semigroup associated
The corresponding semigroup. Fix ααα ∈ (0,1)n and let {P ααα
}t≥0. Concretely, if X is a vector space, then for every function f : {−1,1}n →
to the process {Xααα
(cid:16)
X and t ≥ 0, we denote by

P ααα
(21)
In view of the above interpretation of {Xααα
}t≥0 by means of a Poisson process, the action of the
semigroup {P ααα
(cid:90)
P ααα
t f =

(cid:88)
}t≥0 can be computed via the identity
(cid:88)

Nt(i) > 0 for i ∈ S and Nt(i) = 0 for i (cid:60) S
−t)
|S|

{−1,1}S
Lemma 8. Fix ααα = (α1, . . . , αn) ∈ (0,1)n and let {P ααα
its generator Lααα on a function f : {−1,1}n → X, where X is a vector space, is given by

f d
}t≥0 be the semigroup (21). Then, the action of

t f (x) = E(cid:104)

∀ x ∈ {−1,1}n,

(cid:17)(cid:12)(cid:12)(cid:12) Xααα

(cid:89) i

S⊆{1,...,n}

S⊆{1,...,n}

µα(x)pα

(1− e

−t(n−|S|)

P(cid:110)

∈S

=

e

t

t

t

t

∀ x, y ∈ {−1,1},

(23)

(cid:3)

(cid:128)αi
i f (x)

P ααα
t f (x).

i=1

Lαααf (x) = − n(cid:88)
(cid:12)
(cid:12)
(cid:12)
(cid:12)t=0

d d

Lαααf (x) =

t

t

∀ x ∈ {−1,1}n,

∀ x ∈ {−1,1}n,

i f (x) = f (x)−(cid:82)

where (cid:128)β

{−1,1} f (x1, . . . , xi−1, y, xi+1, . . . , xn) dµβ(y) for β ∈ (0,1).

Proof. The claim follows from the expression (22) of the semigroup and the definition

2.2. Topology. Apart from the probabilistic elements from analysis on biased hypercubes, the
proof of Theorem 1 also has a crucial topological component, following an idea of [56].

6

where wS(a) =(cid:81)

The Borsuk–Ulam theorem. While the Poincaré-type inequality of Enflo for X-valued functions
on {−1,1}n cannot capture the dimension of the target space X, a key part of the argument
towards Theorem 1 is to show that there exists a dim(X)-dimensional subcube of {−1,1}n along
with a bias vector ααα for which the ααα-biased Poincaré inequalities (see Theorem 14 below) on
this subcube and its antipodal yield much better distortion lower bounds. This will be proven
using the Borsuk–Ulam theorem from algebraic topology, see [43].
Theorem 9 (Borsuk–Ulam). For every continuous function g : Sd → Rd, where d ∈ N, there exists
a point w ∈ Sd such that g(w) = g(−w).
Multilinear extension and low-dimensional faces of the cube. Every function f : {−1,1}n → X
(cid:19)
admits a unique multilinear extension on the solid cube [−1,1]n, given by

(cid:88)

(cid:88)

(cid:18) 1

∀ y ∈ [−1,1]n,

F(y) def=

f (x)wS(x)

wS(y),

(24)

S⊆{−1,1}n

2n

x∈{−1,1}n

(cid:110)

Cn
d =

i∈S ai, which is usually referred to as the Fourier–Walsh expansion of f . Ex-
tending f to the continuous cube allows for the use of topological methods. In what follows,
we will exploit the fact that the cube [−1,1]n is equipped with a canonical CW complex struc-
ture. Concretely, for d ∈ {1, . . . , n}, consider the subsets

x ∈ [−1,1]n : there exists σ ⊆ {1, . . . , n} with |σ| ≥ n− d and |x(i)| = 1, ∀ i ∈ σ
n = [−1,1]n and Cn

consisting of all (cid:96)-dimensional faces of [−1,1]n for (cid:96) ≤ d, so that Cn
We shall use the following elementary topological fact (see [56, Lemma 1]).
Lemma 10. If d < n, there exists a continuous map hd : Sd → Cn
Combining this and the Borsuk–Ulam theorem, we deduce the following useful lemma.

(cid:111)
0 = {−1,1}n.
d with hd(−x) = −hd(x), ∀ x ∈ Sd.

(25)

Lemma 11. If n, d ∈ N with d < n, then for every continuous function F : Cn
point z ∈ Cn

d such that F(z)=F(-z).

d

→ Rd there exists a

def= F ◦ hd : Sd → Rd, where hd is the function of Lemma 10. By

Proof. Consider the function g
the Borsuk–Ulam theorem and the oddness of hd, there exists a point w ∈ Sd such that
and the conclusion follows by choosing z = hd(w) ∈ Cn
d.

F(hd(w)) = g(w) = g(−w) = F(hd(−w)) = F(−hd(w))

(26)
(cid:3)

3. Proof of Theorem 1

(cid:32)

We are now ready to proceed to the main part of the proof. The main analytic component is
a biased version of the key formula of [27] for the time derivative of the heat flow on {−1,1}n.

t (·,·; θ) given by
Given t ≥ 0, α ∈ (0,1) and an auxiliary parameter θ ∈ R, consider the matrix ηα





 .
t (·,·; θ).

(cid:33)
t (1,−1; θ)
ηα
ηα
t (1,1; θ)
t (−1,−1; θ)
t (−1,1; θ) ηα
ηα

For future reference, we record the following straightforward properties of ηα
Lemma 12. Fix t ≥ 0 and α ∈ (0,1). Then,
pα
t (x,1)ηα

t (−1, x; θ) = 0

−θ
t (−1,1)
pα
t (−1,−1)
θ
pα

−t−θ
pα
t (1,1)
θ−e
−t
t (1,−1)
pα

t (1, x; θ) + pα







 e

∀ t ≥ 0,

(27)

=

∀ x ∈ {−1,1}, θ ∈ R,
(cid:26)

pα
t (x,1)ηα

t (1, x; θ)2 + pα

and

min
θ∈R

max
x∈{−1,1}

t (x,−1)ηα
(cid:27)

t (x,−1)ηα
(cid:112)

(et − 1)(

t (−1, x; θ)2
−t
t (−1,−1) +

e

αpα
7

=

(cid:112)

(1− α)pα

t (1,1))2

≤ 1
et − 1

.

(28)

(29)

t (1,1)

−t

e

(cid:90)
(cid:90) ∞

0

(cid:88) i

=1

n

(cid:26)

(cid:27)
Proof. The centering condition (28) can be checked easily using the explicit formulas (17) and
(27) of the matrices. For (29), we compute that for any θ ∈ R,
t (x,−1)ηα
t (−1, x; θ)2
(cid:26)
−t − θ)2
(e

t (1, x; θ)2 + pα

pα
t (x,1)ηα

max
x∈{−1,1}

(30)

(cid:27)

t (1,−1)

,

θ2
t (−1,1)pα
t (−1,−1)
pα

,

.

As this is the maximum of two quadratic functions in θ, its minimum is attained at the point
∗
θ

where they intersect in the interval (0, e

= max

t (1,1)pα
pα
−t), namely at
(cid:112)
t (−1,−1)
(1− α)pα

−t(cid:112)
αpα
t (−1,−1) +

e

(cid:112)

∗

θ

=

αpα

.

(31)

The first equality in (29) is immediate, whereas for the inequality we compute

(cid:112)

(cid:112)

−t
t (−1,−1) +

e

≤

αpα

(et − 1)(

(1− α)pα

t (−1,−1) + (1− α)pα
(et − 1)(αpα
(cid:16)
−t
e
−t)(α2 + (1− α)2)
1− (1− e
where both inequalities follow from the convexity of x (cid:55)→ x2.

(et − 1)

t (1,1))2

=

t (1,1))

(cid:17) ≤ 1
et − 1

(32)

,

(cid:3)

(cid:16)

∀ x ∈ {−1,1}n,

(cid:20) n(cid:88)
t f (x) = −E

The key technical ingredient in the proof of Theorem 1 is the following identity.
Proposition 13. Fix n ∈ N, ααα = (α1, . . . , αn) ∈ (0,1)n, t ≥ 0 and θ1, . . . , θn ∈ R. Then, for every
function f : {−1,1}n → X, where X is a vector space, we have

(cid:17)
(cid:128)αi
i f (Xααα
t )
(33)
t (i); θi
}t≥0, it suffices to check the
Proof. In view of (23) and the product structure of the process {Xααα
claim for n = 1, namely that for every β ∈ (0,1), θ ∈ R and f : {−1,1} → X,
t ; θ)(cid:128)βf (Xβ
t )
where the first equality follows from the probabilistic representation (22). Taking into account
that β(cid:128)βf (1) + (1− β)(cid:128)βf (−1) = 0, this amounts to the system of equations

t (cid:128)βf (x) = E(cid:104)

−t(cid:128)βf (x) = P β

(cid:12)
(cid:12)
(cid:12)
(cid:12) Xααα

(cid:12)
(cid:12)
(cid:12) Xβ

ηβ
t (x, Xβ

x(i), Xααα

LαααP ααα

0 = x

0 = x

(34)

ηαi
t

(cid:21)

(cid:105)

i=1

e

,

.

t

−t = pβ
−t = − 1−β

t (1,1; θ)− β
t (−1,1)ηβ

t (1,1)ηβ
β pβ

t (1,−1)ηβ

t (1,−1; θ)

1−β pβ

e

t (−1,−1)ηβ

t (−1,1; θ) + pβ
(cid:3)
which can be easily verified by direct computation.
Theorem 14. Fix p ∈ [1,2] and let (X,(cid:107)·(cid:107)X) be a normed space of Rademacher type p. Then, for any
n ∈ N and ααα = (α1, . . . , αn) ∈ (0,1)n, every function f : {−1,1}n → X satisfies
(cid:13)
(cid:13)
(cid:13)(cid:128)αi

dµααα(x) ≤(cid:16)

t (−1,−1; θ)

(cid:13)
(cid:13)
(cid:13)
(cid:13)f (x)−

X dµααα(x).

2πTp(X)

(cid:13)
(cid:13)
(cid:13)
(cid:13)p

(cid:13)
(cid:13)
(cid:13)p

i f (x)

f dµααα

(cid:17)p

(cid:90)

(cid:90)

(35)

(36)

X

{−1,1}n

{−1,1}n

∀ x ∈ {−1,1},






e

{−1,1}n

(cid:90)

Proof. Writing

f (x)−

f dµααα = P ααα

0 f (x)− P ααα∞f (x) = −

{−1,1}n

8

LαααP ααα

t f (x) dt

(37)



(cid:90)
and using Jensen’s inequality and Proposition 13, we see that for θ1(t), . . . , θn(t) ∈ R,








1/p





1/p






1/p ≤


(cid:90)




(cid:17)

(cid:128)αi
i f (Xααα
t )

(cid:90) ∞

t (i); θi(t)

(cid:13)
(cid:13)
(cid:13)
(cid:13)p
(cid:21)(cid:13)(cid:13)(cid:13)(cid:13)(cid:13)(cid:13)p

x(i), Xααα

dµααα(x)

dµααα(x)

dµααα(x)

{−1,1}n

{−1,1}n

t f (x)

f dµααα

0 = x

ηαi
t

{−1,1}n

(cid:16)

=

X

0

dt

X

{−1,1}n

(cid:90)
(cid:13)
(cid:13)
(cid:13)
(cid:13)f (x)−

(cid:90) ∞
(cid:90)





(cid:90) ∞
(cid:13)
(cid:13) n(cid:88)

(cid:13)

(cid:13)


E

≤

0

0

i=1

X

(cid:13)
(cid:13)
(cid:13)
(cid:13)p
(cid:13)
(cid:20) n(cid:88)
(cid:13)
(cid:13)
(cid:13)
(cid:13)
(cid:13)E
(cid:16)

i=1

(cid:13)
(cid:13)
(cid:13)
(cid:13)LαααP ααα
(cid:12)
(cid:12)
(cid:12)
(cid:12) Xααα





1/p

dt,

(cid:13)
(cid:13)
(cid:13)
(cid:13)p

X

ηαi
t

Xααα
0 (i), Xααα

t (i); θi(t)

(cid:128)αi
i f (Xααα
t )

(cid:17)

dt

(38)

where in the last expectation Xααα
the chain, this expectation can be written as

0 is distributed according to µααα. Now, by the reversibility of

0
which is precisely the desired estimate.

Equipped with the biased Poincaré inequality of Theorem 14, we can conclude the proof.
Proof of Theorem 1. Let X = (Rd,(cid:107) · (cid:107)X) be a d-dimensional normed space and suppose that
f : {−1,1}n → X is a function such that
∀ x, y ∈ {−1,1}n,
(42)
for some D ≥ 1. The conclusion of the theorem follows from [27] when d ≥ n so we shall
assume that d < n. Let F : [−1,1]n → X be the multilinear extension of f given by (24). Then, F
is clearly continuous as a polynomial and therefore, by Lemma 11, there exists a point z ∈ Cn
such that F(z) = F(−z). As z has at least n− d coordinates equal to 1 in absolute value we shall

ρ(x, y) ≤ (cid:107)f (x)− f (y)(cid:107)X ≤ Dρ(x, y)

d

9

dt√
et − 1

,

(41)

(cid:3)

X

X

i=1

i f (x)

X dµααα(x),

(cid:17)

(cid:16)

(39)

i f (x)

dµααα(x)

dµααα(x).

(cid:13)
(cid:13)
(cid:13)p

(cid:128)αi
i f (x)

X dµααα(x)

y(i), x(i); θi(t)

y(i), x(i); θi(t)

y(i), x(i); θi(t)

(cid:13)
(cid:13) n(cid:88)
(cid:13)
(cid:13)

(cid:13)
(cid:13)
(cid:13)
(cid:13)p
ηαi
t
Fixing x ∈ {−1,1}n, equation (28) asserts that each ηαi
t (y(i), x(i); θi(t)) is a centered random
t (x,·) is a product
variable when y(i) is distributed according to pαi
(cid:90)
measure, the Rademacher type condition for sums of centered independent random vectors
(cid:13)
(cid:13)
(cid:13)
(see [33, Proposition 9.11]) yields the bound
(cid:13)p
≤(cid:16)
≤(cid:16)

(cid:13)
(cid:13) n(cid:88)
(cid:13)
(cid:13)
(cid:88)
(cid:18) (cid:88)

(cid:17)(cid:12)(cid:12)(cid:12)p(cid:13)(cid:13)(cid:13)(cid:128)αi
(cid:13)
(cid:13)
(cid:13)p
(cid:17)(cid:12)(cid:12)(cid:12)2(cid:19)p/2(cid:13)(cid:13)(cid:13)(cid:128)αi

(cid:17)
(cid:17)(cid:12)(cid:12)(cid:12)ηαi
(cid:16)
(cid:17)(cid:12)(cid:12)(cid:12)ηαi

t (x(i),·). Therefore, as pααα

(cid:88)
(cid:90)
(cid:17)p
(cid:17)p

(cid:88) i

y(i), x(i); θi(t)

{−1,1}n
n

Xααα
0 (i), Xααα

y(i)∈{−1,1}

t (i);θi(t)

(cid:128)αi
i f (x)

y∈{−1,1}n

y∈{−1,1}n

(cid:128)αi
i f (Xααα
t )

pααα
t (x, y)

pααα
t (x, y)

2Tp(X)

2Tp(X)

x(i), y(i)

x(i), y(i)

{−1,1}n

{−1,1}n

(cid:90)

ηαi
t

ηαi
t

pαi
t

(cid:16)

(cid:16)

(cid:16)

n

=1

(cid:16)

pαi
t

t

t

i=1

{−1,1}n

y(i)∈{−1,1}

(cid:88) i

=1

(40)
where we also used that p ≤ 2. Now, choosing the θi(t) which minimize the quantity in the
left-hand side of (29) with bias αi, and combining (38), (39) and (40), we conclude that


(cid:90)





{−1,1}n

(cid:90)

(cid:13)
(cid:13)
(cid:13)
(cid:13)f (x)−

f dµααα

dµααα(x)

(cid:13)
(cid:13)
(cid:13)
(cid:13)p

(cid:90) ∞
 n(cid:88)





X

i=1

(cid:90)






1/p
(cid:13)
(cid:13)
(cid:13)(cid:128)αi

{−1,1}n
≤ 2Tp(X)

(cid:13)
(cid:13)
(cid:13)p






1/p

{−1,1}n

i f (x)

X dµααα(x)

(cid:16)

(cid:13)
(cid:13) n(cid:88)
(cid:13)
(cid:13)

i=1

E

(cid:17)

(cid:90)

=

(cid:13)
(cid:13)
(cid:13)
(cid:13)p
(cid:88)

X

assume without loss of generality that |z(d + 1)| = . . . = |z(n)| = 1 and consider the functions
h+, h− : {−1,1}d → X which are defined as
(cid:16) 1+z(1)
h±(x) = f

(cid:16)± x(1), . . . ,±x(d),±z(d + 1), . . . ,±z(n)
(cid:17)

(cid:17) ∈ (0,1)d and notice that, by the multilin-
(cid:90)

∀ x ∈ {−1,1}d,
(cid:90)

Consider also the bias vector αααz =
earity of F, we have the identity

, . . . , 1+z(d)

(43)

2

2

.

h−(x) dµαααz(x).

X dµαααz(x)

(cid:13)
(cid:13)
(cid:13)p

1+z(i)

2

i

h−(x)

X dµαααz(x).

=1

Now, in view of the lower Lipschitz condition (42), we clearly have

(cid:13)
(cid:13)
(cid:13)
(cid:13)
(cid:13)h+(x)−h−(x)
(cid:13)X =
(cid:90)
(cid:13)
(cid:13)
for every x ∈ {−1,1}d. On the other hand, for a fixed i ∈ {1, . . . , d} and β = 1+z(i)
(cid:13)
(cid:13)
(cid:13)(cid:128)β
(cid:13)p
X dµβ(x(i)) = β(cid:107)(cid:128)β
(cid:16)
β(1− β)p + (1− β)βp(cid:17)(cid:13)(cid:13)(cid:13)h+(x(1), . . . ,1, . . . , x(d))− h+(x(1), . . . ,−1, . . . , x(d))

(cid:16)−x(1), . . . ,−x(d),−z(d+1), . . . ,−z(n)
(cid:17)(cid:13)(cid:13)(cid:13)X

i h+(x(1), . . . ,1, . . . , x(d))(cid:107)p

X + (1− β)(cid:107)(cid:128)β

i h+(x(1), . . . ,−1, . . . , x(d))(cid:107)p

where in the last equality we used that p ≤ 2 along with the upper Lipschitz condition (42).
The same bound also holds for h−. Integrating the last two inequalities and combining them
with (45), we deduce that

≤ Dp
2p ,

, we have

(cid:13)
(cid:13)
(cid:13)p

i h+(x)

{−1,1}

=

X

2

X

(cid:13)
(cid:13)
(cid:13)
(cid:13)
(cid:13)p
(cid:13)h−(x)− F(−z)
(cid:13)
(cid:13)
(cid:13)
(cid:13)
(cid:13)(cid:128)
(cid:13)p

h+(x)

1+z(i)

X +

2

i

d

X +

{−1,1}d

(cid:13)
(cid:13)
(cid:13)p

(cid:88) i

h+(x) dµαααz(x) = F(z) = F(−z) =
(cid:13)
(cid:13)
(cid:13)
(cid:13)
(cid:13)h+(x)− h−(x)
(cid:13)p
(cid:90)
≤ 2p−1
≤ 22p−1(cid:16)
(cid:16)
(cid:13)
(cid:13)
(cid:13)f

(cid:13)
(cid:13)
(cid:13)h+(x)− F(z)
(cid:90)
(cid:17)p

(cid:13)
(cid:13)
(cid:13)(cid:128)
(cid:17)−f

x(1), . . . , x(d), z(d+1), . . . , z(n)

X dµαααz(x)

πTp(X)

{−1,1}d

{−1,1}d
Therefore, by the triangle inequality and Theorem 14 we get

{−1,1}d

(cid:90)

{−1,1}d

np ≤ (2πTp(X))pdDp,

(44)

(45)

≥ n

(46)
(cid:3)

(47)

which completes the proof of the theorem.
Remark 15. The identity of Proposition 13 in the case of the uniform measure σn (which was ob-
tained in [27]) is simpler. Let ξ1(t), . . . , ξn(t) be i.i.d. random variables distributed according to µβ(t),
2 . Then, for any point x ∈ {−1,1}n, the corresponding unbiased process {Xt(i)}t≥0
where β(t) = 1+e
with X0 = x has distribution equal to x(i)ξi(t) at time t. Thus applying formula (33) with αi = 1
2
and θi(t) = e

−t
2 , we recover the usual identity

−t

(cid:20) n(cid:88)
LPtf (x) = −E

∀ x ∈ {−1,1}n,

ξi(t)− e
−t
et − e
−t
where xξ(t) = (x(1)ξ1(t), . . . , x(n)ξn(t)), as was proven in [27].
4. Proof of Theorem 2

i=1

(cid:16)

(cid:17)(cid:21)

· (cid:128)if

xξ(t)

,

n

=1

(48)

(cid:88) i

Recall that a normed space (X,(cid:107) · (cid:107)X) has cotype q ∈ [2,∞) with constant C ∈ (0,∞) if for
(cid:13)
(cid:13) n(cid:88)
every n ∈ N and v1, . . . , vn ∈ X, we have
(cid:13)
(cid:13)

(cid:107)vi(cid:107)q
X.
We say that X has finite cotype if it has cotype q for some q ∈ [2,∞).

dσn(x) ≥ 1
Cq

(cid:13)
(cid:13)
(cid:13)
(cid:13)q

{−1,1}n

(cid:90)

xivi

i=1

X

10

δi(cid:128)α
i f

Lp(µn

α;X)

dσn(δ)

.

(49)

(cid:13)
(cid:13)
(cid:13)
(cid:13)p
(cid:13)
(cid:13)
(cid:13)
(cid:13)p







1/p






1/p

.

δi(cid:128)α
i f

dσn(δ)

Lp(µn

α;X)

(cid:82)

≤ 2n(cid:107)f (cid:107)Lp(µααα;X),

(50)

(51)

(52)

The proof of Theorem 2 is similar to the arguments of [27, 16] using as input the new iden-
}t≥0. In view of that, we shall omit
tity (33) for the time derivative of the biased semigroup {P ααα
various simple details and we will be less attentive with the values of the implicit constants.
We start by proving the (weaker) biased Pisier inequality, in which the Orlicz norm on the
left hand side of the conclusions of Theorem 2 is replaced by an Lp norm.
Theorem 16. For every α ∈ (0,1), there exists Kα ∈ (0,∞) such that the following holds when
p ∈ [1,∞). For any normed space (X,(cid:107)·(cid:107)X) and any n ∈ N, every function f : {−1,1}n → X satisfies

t

where the last inequality follows from the definitions of (cid:128)α
∀ t ≥ 0,

−tL

≤

∞

α f
α
α

i . Therefore,
(2nt)m

(cid:107)f (cid:107)Lp(µααα;X) = e2nt(cid:107)f (cid:107)Lp(µααα;X).

m!

Since P ααα

t = etL

t f with t = 1

n implies that

1/nf










(cid:90) ∞

Similarly to (38) and (39), we thus have

(cid:13)
(cid:13)
(cid:13)Lp(µααα;X).
(cid:88)
pααα
t (x, y)
y∈{−1,1}n
t (x,·) is a product measure, the centering condition (28) gives
Fix t ≥ 0 and x ∈ {−1,1}n. Since pααα
(cid:88)

(cid:107)f (cid:107)Lp(µααα;X) ≤ e2(cid:13)(cid:13)(cid:13)P ααα
(cid:90)
(cid:13)
(cid:13) n(cid:88)
(cid:13)
(cid:13)
≤ 2p

≤ e2
(cid:88)

(cid:13)
(cid:13) n(cid:88)
(cid:13)
(cid:13)
(cid:17)








1/p

y(i), x(i); θi(t)

y∈{−1,1}n

pααα
t (x, y)

(cid:13)
(cid:13)
(cid:13)
(cid:13)p

(cid:128)α
i f (x)

dµααα(x)

{−1,1}n

(cid:90)

(53)

(54)

(cid:17)

(cid:16)

y(i), x(i); θi(t)

(cid:17)(cid:12)(cid:12)(cid:12)(cid:128)α

dσn(δ).

dt

i=1

i=1

1/n

δi

X

i f (x)

(cid:13)
(cid:13)
(cid:13)
(cid:13)p

y∈{−1,1}n

{−1,1}n
−t
2 and observe that by the contraction principle [33, Theorem 4.4],

i=1

X

t

To prove (49), set θi(t) = e
we can further bound this quantity by

2p (cid:88)

y∈{−1,1}n

pααα
t (x, y)

(cid:90)

(cid:13)
(cid:13) n(cid:88)
(cid:13)
(cid:13)
{−1,1}n
≤ 2p max
χ,ψ∈{−1,1}

i=1

δi

(cid:16)
(cid:12)
(cid:12)
(cid:12)ηα
(cid:16)
(cid:12)
(cid:12)
(cid:12)ηα

t

t

−t
y(i), x(i); e
2

(cid:90)

(cid:17)(cid:12)(cid:12)(cid:12)p

−t
ψ, χ; e
2

11

i f (x)

dσn(δ)

(cid:17)(cid:12)(cid:12)(cid:12)(cid:128)α

(cid:13)
(cid:13)
(cid:13)
(cid:13)p
(cid:13)
(cid:13) n(cid:88)
(cid:13)
(cid:13)

i=1

X

{−1,1}n

δi(cid:128)α

i f (x)

(cid:13)
(cid:13)
(cid:13)
(cid:13)p

X

(55)

dσn(δ).

=0

∞

i=1

i=1

i f

{−1,1}n

{−1,1}n

y(i), x(i); θi(t)

(cid:88) m

(cid:13)
(cid:13) n(cid:88)
(cid:13)
(cid:13)
(cid:13)
(cid:13) n(cid:88)
(cid:13)
(cid:13)

≤ Kp,α(X)
(23)≤ n(cid:88)
(cid:13)
(cid:13)
(cid:13)Lm

(cid:13)
(cid:13)
(cid:13)
(cid:13)
(cid:13)Lp(µn
(cid:13)
(cid:13)
(cid:13)
(cid:13)
(cid:13)Lp(µn
(cid:13)
(cid:13)
(cid:13)
(cid:13)
(cid:13)Lαααf
(cid:13)Lp(µααα;X)
(cid:13)
(cid:13)
(cid:13)Lp(µααα;X)


(cid:90)





≤ Kα(log n + 1)

(cid:90)





(cid:13)
(cid:13)
(cid:13)
(cid:13)
(cid:13)Lp(µααα;X)
(cid:13)(cid:128)α
(cid:13)
(cid:13)
(cid:13)
(cid:13)
(cid:13)e
(cid:13)Lp(µααα;X)
≤
(cid:13)
(cid:107)f (cid:107)Lp(µααα;X) ≤ e2(cid:13)(cid:13)(cid:13)P ααα
(cid:13)
α for t ≥ 0, this inequality applied to P ααα
(cid:13)Lp(µααα;X).
1/nf
(cid:16)
(cid:13)
(cid:13)
(cid:13)
(cid:13)p
(cid:13)
(cid:13) n(cid:88)
(cid:13)
(cid:13)

(cid:88) m

(cid:12)
(cid:12)
(cid:12)ηα

pααα
t (x, y)

(cid:128)α
i f (x)

{−1,1}n

f dµn
α

f dµn
α

tm
m!

(cid:90)

ηα
t

ηα
t

(cid:16)

α;X)

α;X)

α f
α
α

i=1

=0

X

α
α

(cid:90)

(cid:13)
(cid:13)
(cid:13)
(cid:13)
(cid:13)f −

{−1,1}n

(cid:13)
(cid:13)
(cid:13)
(cid:13)
(cid:13)f −

If additionally X is assumed to be of finite cotype, then there exists Kp,α(X) such that

Proof. Let ααα = (α, . . . , α) and without loss of generality assume that
proving the general inequality (49). Notice that, by convexity,

f dµααα = 0. We start by

(cid:16)

i f (x)

{−1,1}n

(cid:33)

i=1

(56)

(57)

t

(cid:16)

δi(cid:128)if

Lp(µααα;X)

{−1,1}n

i=1

pααα
t (x, y)

δi

t

pααα
t (x, y)

δi

t

dt
t

+

1/n

1

0

{−1,1}2n

dσn(δ)

,

−t dt

e

·

(cid:13)
(cid:13)
(cid:13)
(cid:13)p

X

max

χ,ψ∈{−1,1}

−t
ψ, χ; e
2

(cid:32)(cid:90) 1

dσn(δ)dµααα(x)







1/p

−t
y(i), x(i); e
2

(cid:16)
(cid:12)
(cid:12)
(cid:12)ηα
(cid:90) ∞





 1
t ,
−t,
e
(cid:13)
(cid:13) n(cid:88)
(cid:13)
(cid:13)

and thus combining all the above we get

which readily concludes the proof of (49).

It is now elementary to check that we have

For the proof of (50), we combine (38) and (39) with (54) to get

(cid:107)f (cid:107)Lp(µααα;X) (cid:46)α

(cid:90) ∞
(cid:90)








y∈{−1,1}n

(cid:88)
(cid:13)
(cid:13) n(cid:88)
(cid:12)
(cid:13)
(cid:12)
(cid:13)
(cid:12)ηα
(cid:88)
(cid:88)

y∈{−1,1}I(x)

for t ∈ (0,1)
for t ≥ 1
(cid:13)
(cid:13)
(cid:13)
(cid:13)p
(cid:17)(cid:12)(cid:12)(cid:12)(cid:128)α








1/p
(cid:107)f (cid:107)Lp(µααα;X) ≤ 2
dt.
(cid:90)
Fix x ∈ {−1,1}n and t ≥ 0. Denoting by I(x) = {i : x(i) = 1}, the inner term is bounded above by
(cid:88)

(cid:17)(cid:12)(cid:12)(cid:12) (cid:46)α

(cid:90)





(cid:13)
(cid:13) n(cid:88)
(cid:12)
(cid:13)
(cid:12)
(cid:13)
(cid:12)ηα
(cid:17)(cid:12)(cid:12)(cid:12)(cid:128)α
(cid:13)
(cid:13)(cid:88)
(cid:16)
(cid:12)
(cid:13)
(cid:12)
(cid:13)
(cid:12)ηα
(cid:13)
(cid:13)(cid:88)
(cid:16)
(cid:12)
(cid:13)
(cid:12)
(cid:13)
(cid:12)ηα
y(i),−1; e
δi
t (x,·) and δ is a uniformly random sign,
When y ∈ {−1,1}I(x) is distributed according to pααα
2 )|, i ∈ I(x), are independent and identically distributed.
the random variables δi|ηα
(cid:90)
(cid:13)
(cid:13)(cid:88)
Therefore, standard comparison principles going back to works of Maurey and Pisier (see,
(cid:16)
(cid:12)
e.g., [33, Proposition 9.14]) show that if X has cotype q < ∞ and r > max{p, q}, then
(cid:13)
(cid:12)
(cid:13)
(cid:12)ηα
(cid:16)
(cid:16)
(cid:17)(cid:12)(cid:12)(cid:12)ηα
(cid:17)(cid:12)(cid:12)(cid:12)ηα

(cid:88)
(cid:18) (cid:88)
(cid:18) (cid:88)

(cid:17)(cid:12)(cid:12)(cid:12)(cid:128)α
(cid:17)(cid:12)(cid:12)(cid:12)r(cid:19)p/r(cid:90)
(cid:17)(cid:12)(cid:12)(cid:12)r(cid:19)1/r

(cid:17)(cid:12)(cid:12)(cid:12)(cid:128)α
(cid:17)(cid:12)(cid:12)(cid:12)(cid:128)α

−t
2

It is again elementary to show that

y∈{−1,1}n
≤ 2p−1

−t
y(i), x(i); e
2

−t
y(i),1; e
2

−t
y(i),1; e
2

−t
1, ψ; e
2

−t
ψ,1; e
2

y∈{−1,1}I(x)c

t (y(i),1; e

y∈{−1,1}I(x)

(cid:90)
(cid:90)

δi(cid:128)α

i f (x)

dσI(x)c(δ).

(cid:13)
(cid:13)
(cid:13)
(cid:13)p

X

(cid:13)
(cid:13)
(cid:13)
(cid:13)p

X

(cid:13)
(cid:13)
(cid:13)
(cid:13)p

X

(cid:13)
(cid:13)
(cid:13)
(cid:13)p

X

(cid:13)
(cid:13)
(cid:13)
(cid:13)p

X

+ 2p−1

−t
χ, ψ; e
2

pα
t

i f (x)

dσn(δ)

−t
ψ, χ; e
2

t

i f (x)

dσI(x)(δ)

i f (x)

dσI(x)(δ)

dσI(x)(δ).

{−1,1}I(x)c

δi

t

i∈I(x)

δi

t

i∈I(x)

pααα
t (x, y)

pααα
t (x, y)

(cid:46)r,X

pα
t

ψ∈{−1,1}

t

i(cid:60)I(x)

pααα
t (x, y)

{−1,1}I(x)

{−1,1}I(x)

{−1,1}I(x)

−t

t

i∈I(x)

(cid:13)
(cid:13)(cid:88)
(cid:13)
(cid:13)






 1
t1− 1
−t,
e

r

,

for t ∈ (0,1)
for t ≥ 1

.

(59)

(60)

i f (x)

(58)

max
χ∈{−1,1}

ψ∈{−1,1}

(cid:16)

i=1

In view of the convergence of the integrals

combining (59), the corresponding estimate on I(x)c and (58), we finally get

(cid:107)f (cid:107)Lp(µααα;X) (cid:46)r,α,X

δi(cid:128)α

i f (x)

δi(cid:128)α

i f (x)

dσn(δ) dµααα(x)

Now, if F, G are two independent centered X-valued random vectors, then

E(cid:107)F + G(cid:107)p

X

≥ max

X,EG(cid:107)EF[F] + G(cid:107)p

X

= max

X,E(cid:107)G(cid:107)p

X

12

(cid:16)

(cid:90) ∞

e

1

+

(cid:46)r,α

−t dt (cid:46)r 1,
(cid:13)
(cid:13)
(cid:13)(cid:88)
(cid:13)
(cid:13)
(cid:13)
(cid:13)
(cid:13)p

+

X

i(cid:60)I(x)

(cid:111)

(cid:90)










{−1,1}n

r

0

dt
t1− 1

(cid:90) 1
(cid:90)
(cid:13)
(cid:13)(cid:88)
(cid:13)
(cid:13)
(cid:110)EF(cid:107)F + EG[G](cid:107)p

{−1,1}n

i∈I(x)

(cid:13)
(cid:13)
(cid:13)
(cid:13)p
(cid:110)E(cid:107)F(cid:107)p

X








1/p

(61)

.

(62)

(cid:111)

.






1/p
(cid:13)
(cid:13)
(cid:13)
(cid:13)p

δi(cid:128)α

i f (x)

dσn(δ) dµααα(x)


(cid:90)






(cid:46)

{−1,1}n

δi(cid:128)α

i f (x)

Lp(µααα;X)







1/p

dσn(δ)

(63)

(cid:3)

(cid:13)
(cid:13)
(cid:13)
(cid:13)p

X

(cid:13)
(cid:13) n(cid:88)
(cid:13)
(cid:13)

i=1

(cid:18)

Therefore,


(cid:90)





{−1,1}n

(cid:90)

{−1,1}n

(cid:13)
(cid:13)(cid:88)
(cid:13)
(cid:13)

i∈I(x)

(cid:13)
(cid:13)
(cid:13)
(cid:13)p

X

δi(cid:128)α

i f (x)

(cid:13)
(cid:13)(cid:88)
(cid:13)
(cid:13)

i(cid:60)I(x)

+

and this concludes the proof of the theorem.
We now proceed to discuss the vector-valued Lp logarithmic Sobolev inequality of Theorem
2. Recall that given a function f : (Ω, µ) → X, we define the Lp(log L)a Orlicz norm of f as

(cid:40)

(cid:90)

(cid:19)

(cid:41)
dµ(ω) ≤ 1

.

(64)

(cid:107)f (cid:107)Lp(log L)a(µ;X) = inf

γ > 0 :

(cid:107)f (ω)(cid:107)p
γ p

X

loga

Ω

(cid:107)f (ω)(cid:107)p
γ p

X

e +

In the case of the unbiased cube, Theorem 2 was proven recently in [16] and the proof used as
a black box the unbiased analogue of Theorem 16 which is due to [64, 27]. As the modifica-
tions which are required to derive Theorem 2 from Theorem 16 in the biased case are almost
mechanical, we shall only offer a high-level description of the proof.
Sketch of the proof of Theorem 2. Given a scalar-valued function h : {−1,1}n → R, we denote by
(cid:32) n(cid:88)
Mh : {−1,1}n → R+ the asymmetric gradient of h, given by

(65)
(cid:82)
where a+ = max{a,0} for a ∈ R. A combination of the triangle inequality with a technical result
of Talagrand [67, Proposition 5.1] on the biased cube implies that if a vector-valued function
f : {−1,1}n → X satisfies
(cid:107)f (cid:107)

∀ x ∈ {−1,1}n,

α = 0, then

(cid:33)1/2

(cid:128)ih(x)2
+

Mh(x) =

f dµn

(66)

i=1

,

Lp(log L)p/2(µn

α;X) (cid:46)p,α

α;X).

The second term can be controlled by the right hand side of (9) or (10) using the results of
Theorem 16. For the first term, we use the key pointwise inequality of [16], asserting that

(cid:13)
(cid:13)
(cid:13)M(cid:107)f (cid:107)X

(cid:90)





2

(cid:13)
(cid:13)
(cid:13)Lp(µn
α;X) +(cid:107)f (cid:107)L1(µn
(cid:13)
(cid:13)
(cid:13) n(cid:88)
(cid:13)
(cid:13)
(cid:13)
(cid:13)
(cid:13)p

δi(cid:128)if (x)

{−1,1}n

M(cid:107)f (cid:107)X(x) ≤ √
i f (x) ∈(cid:110)

(cid:128)α

i=1

(cid:111)
2α(cid:128)if (x),2(1− α)(cid:128)if (x)






1/p

dσn(δ)

X

.

(67)

∀ x ∈ {−1,1}n,

To further upper bound this by the square function where (cid:128)i is replaced by (cid:128)α

i , observe that

and thus the contraction principle and integration in x yield the desired conclusions.

5. Proof of Theorem 4

In this section we shall prove the equivalence of stable type and metric stable type for norms.
The key ingredient in the proof is the following characterization of stable type (see, e.g., [33,
Proposition 9.12 (iii)]) which goes back at least to [40].
Lemma 17. A Banach space (X,(cid:107)·(cid:107)X) has stable type p ∈ [1,2) if and only if there exists a constant
STp(X) ∈ (0,∞) such that for every n ∈ N and every vectors v1, . . . , vn ∈ X, we have

(cid:90)

(cid:13)
(cid:13) n(cid:88)
(cid:13)
(cid:13)

i=1

{−1,1}n

xivi

(cid:13)
(cid:13)
(cid:13)
(cid:13)p

X

dσn(x) ≤ STp(X)p(cid:13)(cid:13)(cid:13)(cid:16)(cid:107)v1(cid:107)X, . . . ,(cid:107)vn(cid:107)X

(cid:17)(cid:13)(cid:13)(cid:13)p

p,∞.
(cid:96)n

13

(68)
(cid:3)

(69)

Proof of Theorem 4. It is clear from Lemma 17 that any normed space with metric stable type
p also has stable type p. For the converse implication, we treat the case p = 1 separately.
Assuming that X has stable type 1, it follows from [59] that it also has nontrivial Rademacher
type and thus (in view of [57]) finite cotype. Therefore, by the X-valued Pisier inequality with
a dimension-free constant of [27], we get

(cid:90)

{−1,1}n

(cid:13)
(cid:13)
(cid:13)
(cid:13)
(cid:13)X dσn(x) ≤ 2
(cid:13)f (x)− f (−x)

(cid:90)
(cid:90)

{−1,1}n

(cid:13)
(cid:13)
(cid:13)
(cid:13)f (x)−
(cid:90)

(cid:90)
(cid:13)
(cid:13) n(cid:88)
(cid:13)
(cid:13)

{−1,1}n

E

i=1

(cid:46)X

{−1,1}n

{−1,1}n

(cid:13)
(cid:13)
(cid:13)
(cid:13)X
(cid:13)
(cid:13)
(cid:13)
(cid:13)X

f dσn

dσn(x)

δi(cid:128)if (x)

dσn(δ)dσn(x).

Applying Lemma 17 conditionally on x ∈ {−1,1}n, we can further bound this quantity as

(cid:90)

(cid:90)

{−1,1}n

{−1,1}n

(cid:13)
(cid:13) n(cid:88)
(cid:13)
(cid:13)

i=1

E

(cid:13)
(cid:13)
(cid:13)
(cid:13)X
(cid:90)
δi(cid:128)if (x)
≤ ST1(X)

dσn(δ)dσn(x)

(cid:13)(cid:16)(cid:107)(cid:128)1f (x)(cid:107)X, . . . ,(cid:107)(cid:128)nf (x)(cid:107)X
(cid:13)
(cid:13)

(cid:17)(cid:13)(cid:13)(cid:13)(cid:96)n

1,∞ dσn(x)

{−1,1}n

(70)

(71)

−t

and this proves the converse implication since dif (x) = (cid:107)(cid:128)if (x)(cid:107)X.
While this proof extends to all values of p, in the case p > 1 we present a more cumbersome
argument which avoids the X-valued Pisier inequality and thus gives better dependence on
parameters of X. For t ≥ 0, let ξ1(t), . . . , ξn(t) be i.i.d. random variables distributed according
to µβ(t), where β(t) = 1+e
. Then, it follows from the semigroup

(cid:90)

argument leading to (38) along with identity (47) of [27] and Jensen’s inequality that




2 , and denote by ηi(t) = ξi(t)−e
et−e
(cid:90)
(cid:13)
(cid:13)
(cid:13)
(cid:13)
(cid:13)
(cid:13)
(cid:13)
(cid:13)p
(cid:13)f (x)− f (−x)
(cid:13)f (x)−
(cid:13)
(cid:13) n(cid:88)
(cid:13)
(cid:13)



(cid:90)








1/p ≤ 2
(cid:90) ∞
≤ 2






1/p





1/p

(72)

dt

(cid:13)
(cid:13)
(cid:13)
(cid:13)p
(cid:13)
(cid:13)
(cid:13)
(cid:13)p

X


(cid:90)





ηi(t)(cid:128)if (x)

X dσn(x)

dσn(x)

dσn(x)

{−1,1}n

{−1,1}n

{−1,1}n

f dσn

−t
−t

E

X

0

{−1,1}n

Fixing x ∈ {−1,1}n, the independent random vectors η1(t)(cid:128)1f (x), . . . , ηn(t)(cid:128)nf (x) are centered.
(cid:90) ∞
Thus, by standard symmetrization estimates, we can further bound the last term by

i=1

2

0

E

{−1,1}n

ηi(t)(cid:128)if (x)

dσn(x)

dt

{−1,1}n

(cid:13)
(cid:13) n(cid:88)
(cid:13)
(cid:13)

i=1

E

(cid:13)
(cid:13)
(cid:13)
(cid:13)p

X

dσn(x)

δiηi(t)(cid:128)if (x)






1/p

(73)

dt,

(cid:13)
(cid:13)
(cid:13)
(cid:13)p

X






1/p

(cid:90) ∞
(cid:90)





0

≤ 4

where the expectation on the right hand side is with respect to (δi, ηi(t)), where the δi are
uniformly random signs, independent of the ηi(t). Therefore, conditioning first on the values
of η1(t), . . . , ηn(t) and using Lemma 17 for the integrand, we get

(cid:13)
(cid:13)
(cid:13)
(cid:13)p

X

(cid:13)
(cid:13) n(cid:88)
(cid:13)
(cid:13)
(cid:13)
≤ STp(X)pE

i=1

(cid:13)
(cid:13)
(cid:13)
(cid:13)
(cid:13)Xei
(cid:13)(cid:128)if (x)

(cid:13)
(cid:13)
(cid:13)
(cid:13)
(cid:13)p

p,∞
(cid:96)n

ηi(t)

δiηi(t)(cid:128)if (x)

,

(74)

where {e1, . . . , en} is the standard basis of Rn. It is well-known that, since p > 1, (cid:96)n
p,∞ is isomor-
phic to a normed space up to constants depending only on p. Moreover, it has finite cotype
with a constant independent of n (for instance because it can be realized as a real interpola-
tion space between (cid:96)n
2, see [11, Section 5.3]). Therefore, using again the comparison

1 and (cid:96)n

14


(cid:90)





(cid:13)
(cid:13) n(cid:88)
(cid:13)
(cid:13)

i=1

(cid:13)
(cid:13) n(cid:88)
(cid:13)
(cid:13)

i=1

E

principle [33, Proposition 9.14], we have

(cid:13)
(cid:13) n(cid:88)
(cid:13)
(cid:13)
(cid:13)

i=1

E

ηi(t)

(cid:13)
(cid:13)
(cid:13)
(cid:13)
(cid:13)(cid:128)if (x)
(cid:13)Xei

(cid:13)
(cid:13)
(cid:13)
(cid:13)
(cid:13)p

(cid:13)
(cid:13)
(cid:13) n(cid:88)
(cid:13)
(cid:13)
(cid:13)
(cid:13)
(cid:13)
(cid:13)
(cid:13)
(cid:13)Xei
(cid:13)
(cid:13)p
(cid:46)p (cid:107)η1(t)(cid:107)p
(cid:13)(cid:16)(cid:107)(cid:128)1f (x)(cid:107)X, . . . ,(cid:107)(cid:128)nf (x)(cid:107)X
(cid:13)
(cid:13)
= (cid:107)η1(t)(cid:107)p
is the cotype of (cid:96)n

(cid:13)
(cid:13)
(cid:13)(cid:128)if (x)

p,∞
(cid:96)n

p,∞
(cid:96)n

i=1

δi

E

Lr

Lr

(cid:17)(cid:13)(cid:13)(cid:13)p

p,∞
(cid:96)n

(cid:48)

(cid:48)

, p}, where r

for any r > max{r
p,∞. Combining all the above along with the fact
that t (cid:55)→ (cid:107)η1(t)(cid:107)Lr is integrable for any r < ∞, we conclude the proof. It is worth emphasizing
that this proof shows additionally that the metric stable type constant of X is proportional to
(cid:3)
the parameter STp(X) of (69) up to constants depending only on p, provided that p > 1.

Finally, we present the simple proof of the distortion bound (16).

Proof of Proposition 6. Suppose that (M, dM) has metric stable type p with constant S and sup-
pose that a function f : {−1,1}n → M satisfies

for some constants s > 0 and D ≥ 1. Combined with the stable type assumption, this gives

sp(cid:107)w(cid:107)p

(cid:96)n
1

(76)≤

dM

{−1,1}n

(cid:90)

∀ x, y ∈ {−1,1}n,

(cid:16)
(cid:17)p
f (x), f (−x)
(76)≤ (sSD)p

f (x), f (y)

sρw(x, y) ≤ dM
(cid:90)

dσn(x) ≤ Sp
(cid:90)

(cid:13)(cid:16)
(cid:13)
(cid:13)
(cid:107)(w1, . . . , wn)(cid:107)p

{−1,1}n

{−1,1}n

(cid:16)

(cid:17) ≤ sDρw(x, y)
(cid:17)(cid:13)(cid:13)(cid:13)p

p,∞ dσn(x)
(cid:96)n

d1f (x), . . . , dnf (x)
p,∞ dσn(x) = (sSD)p(cid:107)w(cid:107)p
p,∞.
(cid:96)n

(cid:96)n

(75)

(76)

(77)

(cid:3)

Rearranging gives the desired lower bound (16) for the distortion D.

6. Discussion and open problems

1. The proof of Theorem 1 in fact implies a Poincaré-type inequality for restrictions of functions
f : {−1,1}n → X if dim(X) < n, which in turn yields the refined distortion lower bounds. An
inspection of the argument reveals that for every such f there exists a subset σ ⊆ {1, . . . , n} with
(cid:90)
|σ| ≤ dim(X), a point w ∈ {−1,1}σ c and a bias vector ααα = (αi)i∈σ ∈ (0,1)σ such that
(cid:13)
(cid:13)
(cid:13)p

(cid:13)
(cid:13)
(cid:13)f (x, w)−f (−x,−w)
≤ 22p−1(cid:16)

(cid:13)
(cid:13)
(cid:13)(cid:128)αi
i f (−x,−w)

(cid:17)p(cid:88)

(cid:13)
(cid:13)
(cid:13)(cid:128)αi

X dµααα(x)

i f (x, w)

πTp(X)

(cid:13)
(cid:13)
(cid:13)p

(cid:13)
(cid:13)
(cid:13)p

{−1,1}σ

(cid:90)

(78)

X dµααα(x).

X +

i∈σ

{−1,1}σ

2. Such refinements of Poincaré-type inequalities for topological reasons had not been ex-
ploited since Oleszkiewicz’s original work [56]. The last decades have seen the development of
many metric inequalities on graphs which yield nonembeddability results into normed spaces.
We believe that investigating whether the distortion estimates which one obtains this way can
be further improved assuming upper bounds for the dimension of the target space is a very
worthwhile research program. As examples, we mention the nonembeddability of graphs with
large girth into uniformly smooth spaces [39, 51], of (cid:96)∞-grids into spaces of finite cotype [45]
and of trees [12, 36] and diamond graphs [31, 23] into uniformly convex spaces.
3. The results of [27] in fact imply that any Lipschitz embedding of {−1,1}n into a normed space
−1n1−1/p.
of Rademacher type p incurs p-average distortion at least a constant multiple of Tp(X)
It would be interesting to understand whether the bound of Theorem 1 can be extended to
average distortion embeddings beyond bi-Lipschitz ones.
4. The Poincaré inequality (78) implies that any f : {−1,1}n → X = (Rd,(cid:107)·(cid:107)X) satisfies

sup

(x,y): ρ(x,y)=n

n

(cid:107)f (x)− f (y)(cid:107)X

sup

(x,y): ρ(x,y)=1

15

(cid:107)f (x)− f (y)(cid:107)X (cid:38)

n

Tp(X)d1/p ,

(79)

(80)

(81)

(cid:19)
p , where p ∈ [1,2], given by
xi
(cid:12)p(cid:33)1/p
(cid:12)
(cid:12)
(cid:12)

= d1/p

(cid:88) i

∈Id

xi

d . Then, the mapping f : {−1,1}n → (cid:96)d
∀ x ∈ {−1,1}n,
xi, . . . ,
(cid:13)
(cid:13)
(cid:13)
(cid:13)
(cid:13)f (x)− f (y)
(cid:13)(cid:96)d

(cid:18)(cid:88)
(cid:32) d(cid:88)

(cid:12)
(cid:12)(cid:88)
(cid:12)
(cid:12)

f (x) def=

= inf

i∈I1

p

x∈{−1,1}n

i∈Ik

k=1

provided that d < n. This is stronger than the lower bound of Theorem 1 as it implies that f
incurs large distortion only on specific pairs of points, namely those which are either antipodal
or connected by an edge. Moreover though, (79) is sharp for any value of d and p. Indeed,
assume without loss of generality that n
d is an odd integer and consider a partition I1, . . . , Id of
{1, . . . , n} in d parts of size n

satisfies

inf

(x,y): ρ(x,y)=n

and (cid:107)f (x)− f (y)(cid:107)
5. The functional inequality (78) in fact implies that if θ ∈ (0,1), then the bi-Lipschitz distor-
tion of the θ-snowflake of {−1,1}n into a finite dimensional normed space X satisfies

= 1 when ρ(x, y) = 1.

(cid:96)d
p

(cid:16){−1,1}n, ρθ(cid:17) (cid:38)

cX

nθ

Tp(X)min{n,dim(X)}1/p .

(82)

6. As the reasons behind the impossibility of dimension reduction of Theorem 1 are partly
topological, it is natural to ask in what other settings can one deduce similar conclusions. For
instance, do d-dimensional manifolds of nonpositive or nonnegative curvature (which have
Enflo type 2, see [55]) admit the same distortion bounds as normed spaces of type 2?
7. In this paper we extended the semigroup machinery developed in [27] to the biased cube,
driven by the geometric application obtained in Theorem 1. As an aside, this led to biased
versions of the vector-valued Poincaré and logarithmic Sobolev inequalities of [27, 16]. The
same reasoning also yields biased extensions of the vector-valued versions of Talagrand’s in-
fluence inequality [68] which were studied in [17] for the uniform probability measure on the
hypercube. Moreover, combining the biased semigroup machinery with the arguments of [10,
Section 2.5], one can derive biased versions of the isoperimetric-type inequalities of Eldan and
Gross [18]. As these extensions are mostly mechanical given the material of this paper and in
lack of a specific application which may follow from them, we omit them.

References

[1] Alexandr Andoni, Moses S. Charikar, Ofer Neiman, and Huy L. Nguyen. Near linear lower bound for dimen-
sion reduction in (cid:96)1. In 2011 IEEE 52nd Annual Symposium on Foundations of Computer Science—FOCS 2011,
pages 315–323. IEEE Computer Soc., Los Alamitos, CA, 2011.

[2] Alexandr Andoni, Piotr Indyk, and Ilya Razenshteyn. Approximate nearest neighbor search in high dimen-
sions. In Proceedings of the International Congress of Mathematicians—Rio de Janeiro 2018. Vol. IV. Invited lec-
tures, pages 3287–3318. World Sci. Publ., Hackensack, NJ, 2018.

[3] Juan Arias-de-Reyna and Luis Rodríguez-Piazza. Finite metric spaces needing high dimension for Lipschitz

embeddings in Banach spaces. Israel J. Math., 79(1):103–111, 1992.

[4] Keith Ball. The Ribe programme. Astérisque, (352):Exp. No. 1047, viii, 147–159, 2013. Séminaire Bourbaki.

Vol. 2011/2012. Exposés 1043–1058.

[5] Keith Ball, Eric A. Carlen, and Elliott H. Lieb. Sharp uniform convexity and smoothness inequalities for trace

norms. Invent. Math., 115(3):463–482, 1994.

[6] Imre Bárány and Victor S. Grinberg. On some combinatorial questions in finite-dimensional spaces. Linear

Algebra Appl., 41:1–9, 1981.

[7] Yair Bartal, Lee-Ad Gottlieb, and Ofer Neiman. On the impossibility of dimension reduction for doubling

subsets of (cid:96)p. SIAM J. Discrete Math., 29(3):1207–1222, 2015.

[8] Yair Bartal and Manor Mendel. Dimension reduction for ultrametrics. In Proceedings of the Fifteenth Annual

ACM-SIAM Symposium on Discrete Algorithms, pages 664–665. ACM, New York, 2004.

[9] Florent Baudier, Krzysztof Swieçicki, and Andrew Swift. No dimension reduction for doubling subsets of (cid:96)q

when q > 2 revisited. J. Math. Anal. Appl., 504(2):Paper No. 125407, 18, 2021.

16

[10] David Beltran, Paata Ivanisvili, and José Madrid. On sharp isoperimetric inequalities on the hypercube.

Preprint available at https://arxiv.org/abs/2303.06738, 2023.

[11] Jöran Bergh and Jörgen Löfström. Interpolation spaces. An introduction, volume No. 223 of Grundlehren der

Mathematischen Wissenschaften. Springer-Verlag, Berlin-New York, 1976.

[12] Jean Bourgain. The metrical interpretation of superreflexivity in Banach spaces. Israel J. Math., 56(2):222–230,

1986.

[13] Jean Bourgain, Vitali Milman, and Haim Wolfson. On type of metric spaces. Trans. Amer. Math. Soc.,

294(1):295–317, 1986.

[14] Bo Brinkman and Moses Charikar. On the impossibility of dimension reduction in l1. J. ACM, 52(5):766–788,

2005.

[15] Bo Brinkman, Adriana Karagiozova, and James R. Lee. Vertex cuts, random walks, and dimension reduction in
series-parallel graphs. In STOC’07—Proceedings of the 39th Annual ACM Symposium on Theory of Computing,
pages 621–630. ACM, New York, 2007.

[16] Dario Cordero-Erausquin and Alexandros Eskenazis. Discrete logarithmic Sobolev inequalities in Banach

spaces. Preprint available at https://arxiv.org/abs/2304.03878, 2023.

[17] Dario Cordero-Erausquin and Alexandros Eskenazis. Talagrand’s influence inequality revisited. Anal. PDE,

16(2):571–612, 2023.

[18] Ronen Eldan and Renan Gross. Concentration on the Boolean hypercube via pathwise stochastic analysis.

Invent. Math., 230(3):935–994, 2022.

[19] Per Enflo. On a problem of Smirnov. Ark. Mat., 8:107–109, 1969.
[20] Per Enflo. On the nonexistence of uniform homeomorphisms between Lp-spaces. Ark. Mat., 8:103–105, 1969.
[21] Per Enflo. On infinite-dimensional topological groups. In Séminaire sur la Géométrie des Espaces de Banach

(1977–1978), pages Exp. No. 10–11, 11. École Polytech., Palaiseau, 1978.

[22] Alexandros Eskenazis. On Pisier’s inequality for UMD targets. Canad. Math. Bull., 64(2):282–291, 2021.
[23] Alexandros Eskenazis, Manor Mendel, and Assaf Naor. Diamond convexity: a bifurcation in the Ribe program.

Preprint, 2023.

[24] Anupam Gupta, Robert Krauthgamer, and James R. Lee. Bounded geometries, fractals, and low-distortion
embeddings. In 44th Annual IEEE Symposium on Foundations of Computer Science, 2003. Proceedings., pages
534–543, 2003.

[25] Tuomas Hytönen and Assaf Naor. Pisier’s inequality revisited. Studia Math., 215(3):221–235, 2013.
[26] Piotr Indyk. Algorithmic applications of low-distortion geometric embeddings. In 42nd IEEE Symposium on
Foundations of Computer Science (Las Vegas, NV, 2001), pages 10–33. IEEE Computer Soc., Los Alamitos, CA,
2001.

[27] Paata Ivanisvili, Ramon van Handel, and Alexander Volberg. Rademacher type and Enflo type coincide. Ann.

of Math. (2), 192(2):665–678, 2020.

[28] Robert C. James. Nonreflexive spaces of type 2. Israel J. Math., 30(1-2):1–13, 1978.
[29] William B. Johnson and Joram Lindenstrauss. Extensions of Lipschitz mappings into a Hilbert space. In Con-
ference in modern analysis and probability (New Haven, Conn., 1982), volume 26 of Contemp. Math., pages
189–206. Amer. Math. Soc., Providence, RI, 1984.

[30] William B. Johnson, Joram Lindenstrauss, and Gideon Schechtman. On Lipschitz embedding of finite metric
spaces in low-dimensional normed spaces. In Geometrical aspects of functional analysis (1985/86), volume 1267
of Lecture Notes in Math., pages 177–184. Springer, Berlin, 1987.

[31] William B. Johnson and Gideon Schechtman. Diamond graphs and super-reflexivity. J. Topol. Anal., 1(2):177–

[32] Vincent Lafforgue and Assaf Naor. A doubling subset of Lp for p > 2 that is inherently infinite dimensional.

189, 2009.

Geom. Dedicata, 172:387–398, 2014.

[33] Michel Ledoux and Michel Talagrand. Probability in Banach spaces, volume 23 of Ergebnisse der Mathematik und
ihrer Grenzgebiete (3) [Results in Mathematics and Related Areas (3)]. Springer-Verlag, Berlin, 1991. Isoperimetry
and processes.

[34] James R. Lee, Manor Mendel, and Assaf Naor. Metric structures in L1: dimension, snowflakes, and average

distortion. European J. Combin., 26(8):1180–1190, 2005.

[35] James R. Lee and Assaf Naor. Embedding the diamond graph in Lp and dimension reduction in L1. Geom.

Funct. Anal., 14(4):745–747, 2004.

[36] James R. Lee, Assaf Naor, and Yuval Peres. Trees and Markov convexity. Geom. Funct. Anal., 18(5):1609–1659,

2009.

[37] David A. Levin and Yuval Peres. Markov chains and mixing times. American Mathematical Society, Providence,
RI, second edition, 2017. With contributions by Elizabeth L. Wilmer, With a chapter on “Coupling from the
past” by James G. Propp and David B. Wilson.

[38] Nathan Linial. Finite metric-spaces—combinatorics, geometry and algorithms. In Proceedings of the Interna-

tional Congress of Mathematicians, Vol. III (Beijing, 2002), pages 573–586. Higher Ed. Press, Beijing, 2002.

[39] Nathan Linial, Avner Magen, and Assaf Naor. Girth and Euclidean distortion. Geom. Funct. Anal., 12(2):380–

394, 2002.

17

[40] Michael B. Marcus and Gilles Pisier. Characterizations of almost surely continuous p-stable random Fourier

series and strongly stationary processes. Acta Math., 152(3-4):245–301, 1984.

[41] Jiří Matoušek. Note on bi-Lipschitz embeddings into normed spaces. Comment. Math. Univ. Carolin., 33(1):51–

55, 1992.

[42] Jiří Matoušek. On the distortion required for embedding finite metric spaces into normed spaces. Israel J.

Math., 93:333–344, 1996.

[43] Jiří Matoušek. Using the Borsuk-Ulam theorem. Universitext. Springer-Verlag, Berlin, 2003. Lectures on topo-
logical methods in combinatorics and geometry, Written in cooperation with Anders Björner and Günter M.
Ziegler.

[44] Bernard Maurey and Gilles Pisier. Séries de variables aléatoires vectorielles indépendantes et propriétés

géométriques des espaces de Banach. Studia Math., 58(1):45–90, 1976.

[45] Manor Mendel and Assaf Naor. Metric cotype. Ann. of Math. (2), 168(1):247–298, 2008.
[46] Assaf Naor. An introduction to the Ribe program. Jpn. J. Math., 7(2):167–233, 2012.
[47] Assaf Naor. Comparison of metric spectral gaps. Anal. Geom. Metr. Spaces, 2(1):1–52, 2014.
[48] Assaf Naor. A spectral gap precludes low-dimensional embeddings. In 33rd International Symposium on Com-
putational Geometry, volume 77 of LIPIcs. Leibniz Int. Proc. Inform., pages Art. No. 50, 16. Schloss Dagstuhl.
Leibniz-Zent. Inform., Wadern, 2017.

[49] Assaf Naor. Metric dimension reduction: a snapshot of the Ribe program. In Proceedings of the International
Congress of Mathematicians—Rio de Janeiro 2018. Vol. I. Plenary lectures, pages 759–837. World Sci. Publ.,
Hackensack, NJ, 2018.

[50] Assaf Naor. An average John theorem. Geom. Topol., 25(4):1631–1717, 2021.
[51] Assaf Naor, Yuval Peres, Oded Schramm, and Scott Sheffield. Markov chains in smooth Banach spaces and

Gromov-hyperbolic metric spaces. Duke Math. J., 134(1):165–197, 2006.

[52] Assaf Naor, Gilles Pisier, and Gideon Schechtman. Impossibility of dimension reduction in the nuclear norm.

Discrete Comput. Geom., 63(2):319–345, 2020.

[53] Assaf Naor and Gideon Schechtman. Remarks on non linear type and Pisier’s inequality. J. Reine Angew. Math.,

552:213–236, 2002.

[54] Assaf Naor and Robert Young. Foliated corona decompositions. Acta Math., 229(1):55–200, 2022.
[55] Shin-Ichi Ohta. Markov type of Alexandrov spaces of non-negative curvature. Mathematika, 55(1-2):177–189,

2009.

[56] Krzysztof Oleszkiewicz. On a discrete version of the antipodal theorem. Fund. Math., 151(2):189–194, 1996.
[57] Gilles Pisier. Sur les espaces de Banach qui ne contiennent pas uniformément de l1
n. C. R. Acad. Sci. Paris Sér.

A-B, 277:A991–A994, 1973.

[58] Gilles Pisier. “Type” des espaces normés. In Séminaire Maurey-Schwartz 1973–1974: Espaces Lp, applications

radonifiantes et géométrie des espaces de Banach, Exp. No. 3,, pages 12 pp. (errata, p. E.1). „ 1974.

[59] Gilles Pisier. Une propriété du type p-stable. In Séminaire Maurey-Schwartz 1973–1974: Espaces Lp, applica-
tions radonifiantes et géométrie des espaces de Banach, pages Exp. No. 8, 10 pp. (errata, p. E.1). École Polytech.,
Paris, 1974.

[60] Gilles Pisier. Martingales with values in uniformly convex spaces. Israel J. Math., 20(3-4):326–350, 1975.
[61] Gilles Pisier. Un exemple concernant la super-réflexivité. In Séminaire Maurey-Schwartz 1974–1975: Espaces

Lp applications radonifiantes et géométrie des espaces de Banach, Annexe No. 2,, page 12. „ 1975.

1982.

[63] Gilles Pisier. On the dimension of the ln

[62] Gilles Pisier. Holomorphic semigroups and the geometry of Banach spaces. Ann. of Math. (2), 115(2):375–392,
p -subspaces of Banach spaces, for 1 ≤ p < 2. Trans. Amer. Math. Soc.,
[64] Gilles Pisier. Probabilistic methods in the geometry of Banach spaces. In Probability and analysis (Varenna,

276(1):201–211, 1983.

1985), volume 1206 of Lecture Notes in Math., pages 167–241. Springer, Berlin, 1986.

[65] Gilles Pisier and Quanhua Xu. Random series in the real interpolation spaces between the spaces vp. In
Geometrical aspects of functional analysis (1985/86), volume 1267 of Lecture Notes in Math., pages 185–209.
Springer, Berlin, 1987.

[66] Martin Ribe. On uniformly homeomorphic normed spaces. Ark. Mat., 14(2):237–244, 1976.
[67] Michel Talagrand. Isoperimetry, logarithmic Sobolev inequalities on the discrete cube, and Margulis’ graph

connectivity theorem. Geom. Funct. Anal., 3(3):295–314, 1993.

[68] Michel Talagrand. On Russo’s approximate zero-one law. Ann. Probab., 22(3):1576–1587, 1994.
[69] Santosh S. Vempala. The random projection method, volume 65 of DIMACS Series in Discrete Mathematics and
Theoretical Computer Science. American Mathematical Society, Providence, RI, 2004. With a foreword by Chris-
tos H. Papadimitriou.

CNRS, Institut de Mathématiques de Jussieu, Sorbonne Université, France and Trinity College, University

of Cambridge, UK.

Email address: alexandros.eskenazis@imj-prg.fr, ae466@cam.ac.uk

18

